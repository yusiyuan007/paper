% !TeX root = ../main.tex

\chapter{结果与展望}

本章主要用来总结本文的工作,总结本文的创新点和不足,并给出在未来工作中的展望和希望研究的部分。

\section{论文总结}

目前部门内的仿真建模平台是基于SystemC的,模型建模做的十分精密,但因此整个仿真平台也十分复杂,仿真运行比较慢,运行一次用例仿真时间较长。因此,为了解决仿真时间长和仿真建模繁琐的同时,也为了满足后续设计空间探索的要求,本文设计并实现了基于Simpy的轻量级仿真平台,并基于轻量级仿真平台打通了设计空间探索流程。

在前期的调研过程中,我们最终确定了以Simpy为仿真环境,以python为开发语言,这样大大减少了开发时间。以之前基于SystemC的SAE仿真平台各个模块的建模为原型,重新实现了各个模块的建模。实现了硬件模块建模、硬件平台搭建、用例文件解析、任务调度以及数据库输出的仿真平台功能。并兼容原有的SAE仿真平台,与原有的仿真平台用例输入和硬件配置文件输入一致,可以实现一个用例在两个仿真平台上运行,这样了降低了后续开发人员的工作量。

在设计空间探索流程中,我们通过调研现在业内设计空间探索以及多目标优化的案例,最终确定了以进化算法为设计空间探索的方法。为了简化进化算法的每一代种群的计算过程,我们采用将仿真过程通过机器学习拟化为一个预测模型,这样将执行一次设计空间探索流程的时间从4天缩短到2分钟,这样研究人员可以将更多的时间却研究设计空间探索的结果。

基于以上方面,本文实现的轻量级仿真平台可以实现业务仿真并大大减少了仿真时间,并在后续一些方案设计中体现了良好的性能。设计空间探索流程的打通为部门内部方案设计提供了新思路和新方法。

\section{问题和展望}

本文实现的仿真平台虽然能够对比原平台能够实现大部分功能,但调度器部分在原型设计过程中仍十分简陋,资源调度方面不够灵活。有些硬件模块没有实现,定时器功能没有实现,这些都有待后续完善。

在设计空间探索方面,整个流程虽然打通,但是整个流程的精度并没有很高,只是对现有的仿真平台做出一个针对性的探索,并没有对更多的平台进行适配,甚至不仅对仿真平台,还可以对直接从单板上提取的输入输出的格式进行适配。在接下来的完善过程中,可以改进预测模型的部分,可以通过这方面去适配更多的输入格式,通过调整进化算法去改善整个设计空间探索流程的精度
 
